% Options for packages loaded elsewhere
\PassOptionsToPackage{unicode}{hyperref}
\PassOptionsToPackage{hyphens}{url}
%
\documentclass[
]{article}
\usepackage{amsmath,amssymb}
\usepackage{iftex}
\ifPDFTeX
  \usepackage[T1]{fontenc}
  \usepackage[utf8]{inputenc}
  \usepackage{textcomp} % provide euro and other symbols
\else % if luatex or xetex
  \usepackage{unicode-math} % this also loads fontspec
  \defaultfontfeatures{Scale=MatchLowercase}
  \defaultfontfeatures[\rmfamily]{Ligatures=TeX,Scale=1}
\fi
\usepackage{lmodern}
\ifPDFTeX\else
  % xetex/luatex font selection
\fi
% Use upquote if available, for straight quotes in verbatim environments
\IfFileExists{upquote.sty}{\usepackage{upquote}}{}
\IfFileExists{microtype.sty}{% use microtype if available
  \usepackage[]{microtype}
  \UseMicrotypeSet[protrusion]{basicmath} % disable protrusion for tt fonts
}{}
\makeatletter
\@ifundefined{KOMAClassName}{% if non-KOMA class
  \IfFileExists{parskip.sty}{%
    \usepackage{parskip}
  }{% else
    \setlength{\parindent}{0pt}
    \setlength{\parskip}{6pt plus 2pt minus 1pt}}
}{% if KOMA class
  \KOMAoptions{parskip=half}}
\makeatother
\usepackage{xcolor}
\usepackage[margin=1in]{geometry}
\usepackage{color}
\usepackage{fancyvrb}
\newcommand{\VerbBar}{|}
\newcommand{\VERB}{\Verb[commandchars=\\\{\}]}
\DefineVerbatimEnvironment{Highlighting}{Verbatim}{commandchars=\\\{\}}
% Add ',fontsize=\small' for more characters per line
\newenvironment{Shaded}{}{}
\newcommand{\AlertTok}[1]{\textbf{#1}}
\newcommand{\AnnotationTok}[1]{\textit{#1}}
\newcommand{\AttributeTok}[1]{#1}
\newcommand{\BaseNTok}[1]{#1}
\newcommand{\BuiltInTok}[1]{#1}
\newcommand{\CharTok}[1]{#1}
\newcommand{\CommentTok}[1]{\textit{#1}}
\newcommand{\CommentVarTok}[1]{\textit{#1}}
\newcommand{\ConstantTok}[1]{#1}
\newcommand{\ControlFlowTok}[1]{\textbf{#1}}
\newcommand{\DataTypeTok}[1]{\underline{#1}}
\newcommand{\DecValTok}[1]{#1}
\newcommand{\DocumentationTok}[1]{\textit{#1}}
\newcommand{\ErrorTok}[1]{\textbf{#1}}
\newcommand{\ExtensionTok}[1]{#1}
\newcommand{\FloatTok}[1]{#1}
\newcommand{\FunctionTok}[1]{#1}
\newcommand{\ImportTok}[1]{#1}
\newcommand{\InformationTok}[1]{\textit{#1}}
\newcommand{\KeywordTok}[1]{\textbf{#1}}
\newcommand{\NormalTok}[1]{#1}
\newcommand{\OperatorTok}[1]{#1}
\newcommand{\OtherTok}[1]{#1}
\newcommand{\PreprocessorTok}[1]{\textbf{#1}}
\newcommand{\RegionMarkerTok}[1]{#1}
\newcommand{\SpecialCharTok}[1]{#1}
\newcommand{\SpecialStringTok}[1]{#1}
\newcommand{\StringTok}[1]{#1}
\newcommand{\VariableTok}[1]{#1}
\newcommand{\VerbatimStringTok}[1]{#1}
\newcommand{\WarningTok}[1]{\textit{#1}}
\usepackage{graphicx}
\makeatletter
\def\maxwidth{\ifdim\Gin@nat@width>\linewidth\linewidth\else\Gin@nat@width\fi}
\def\maxheight{\ifdim\Gin@nat@height>\textheight\textheight\else\Gin@nat@height\fi}
\makeatother
% Scale images if necessary, so that they will not overflow the page
% margins by default, and it is still possible to overwrite the defaults
% using explicit options in \includegraphics[width, height, ...]{}
\setkeys{Gin}{width=\maxwidth,height=\maxheight,keepaspectratio}
% Set default figure placement to htbp
\makeatletter
\def\fps@figure{htbp}
\makeatother
\setlength{\emergencystretch}{3em} % prevent overfull lines
\providecommand{\tightlist}{%
  \setlength{\itemsep}{0pt}\setlength{\parskip}{0pt}}
\setcounter{secnumdepth}{-\maxdimen} % remove section numbering
\usepackage{fancyhdr}
\usepackage[absolute]{textpos}
\pagestyle{fancy}
 
\fancyhead[L]{
    \begin{textblock*}{1.5in}[0.365,0.40](1.5in,1in)
            test1
    \end{textblock*}
    \begin{textblock*}{5.375in}(2.5in,0.85in)   % 6.375=8.5 - 1.5 - 0.625
        test2
    \end{textblock*}
    \begin{textblock*}{5.375in}(4.4in,0.85in)   % 6.375=8.5 - 1.5 - 0.625
        test3
    \end{textblock*}
    \begin{textblock*}{5.375in}(6.1in,0.85in)   % 6.375=8.5 - 1.5 - 0.625
        test4
    \end{textblock*}
}
\ifLuaTeX
  \usepackage{selnolig}  % disable illegal ligatures
\fi
\IfFileExists{bookmark.sty}{\usepackage{bookmark}}{\usepackage{hyperref}}
\IfFileExists{xurl.sty}{\usepackage{xurl}}{} % add URL line breaks if available
\urlstyle{same}
\hypersetup{
  pdftitle={Exercice 1},
  pdfauthor={Samuel Orso},
  hidelinks,
  pdfcreator={LaTeX via pandoc}}

\title{Exercice 1}
\author{Samuel Orso}
\date{2024-08-25}

\begin{document}
\maketitle

\hypertarget{lecture-2-r-markdown}{%
\subsection{Lecture \#2: R Markdown}\label{lecture-2-r-markdown}}

Basic manipulations:\\
1. Create a RMarkdown HTML document in \texttt{RStudio} and
``\texttt{knit}'' it. 1. Create a new header of type 2. 1. Make a linear
regression with ``Sepal Length'' as a response and ``Sepal Width'' as an
explanatory variable from the \texttt{iris} dataset and save the result.
1. Highlight the code with \texttt{monochrome} style. 1. Print the
summary of the linear regression. 1. Include the QQplot from the linear
regression. Change to filled dots. 1. Print the head of the
\texttt{iris} dataset with \texttt{kable}. 1. Remove the \texttt{.} from
the labels (click
\href{https://bookdown.org/yihui/rmarkdown-cookbook/kable.html\#change-column-names}{here}).

More advanced manipulations: 1. Install \texttt{kableExtra}. And perform
the examples shown in the slides with \texttt{iris} dataset. 1. Using
Mathpix, reproduce equation (6.1) of the paper
\url{https://arxiv.org/abs/math/0303109} 1. Add the Reference and cite
it in the RMarkdown. 1. Recreate your RMarkdown into a Quarto document.

\hypertarget{lecture-3-github}{%
\subsection{Lecture \#3: GitHub}\label{lecture-3-github}}

\begin{enumerate}
\def\labelenumi{\arabic{enumi}.}
\tightlist
\item
  Create a GitHub repo for the RMarkdown file (.Rmd) you created in the
  previous exercise.
\item
  Edit the README.md file, push the .Rmd.
\item
  By two. Invite (person A) someone else (person B) to work on your repo
  and try:
\end{enumerate}

\begin{itemize}
\tightlist
\item
  Repo is up-to-date. Person B modifies .Rmd and pushes the changes,
  person A pulls the changes.
\item
  Repo is up-to-date. Person A modifies 1st section of .Rmd, person B
  modifies 2nd section (no conflict) of .Rmd. No push, no pull in
  between. Now person A commits and pushes. Then person B tries to
  commit and push. Try to solve until repo is up-to-date.
\item
  Same as last point, but person B modifies 1st section of .Rmd
  (conflict).
\end{itemize}

\hypertarget{lecture-4-data-structures}{%
\subsection{Lecture \#4: Data
structures}\label{lecture-4-data-structures}}

Using the following code:

\begin{Shaded}
\begin{Highlighting}[]
\FunctionTok{set.seed}\NormalTok{(}\DecValTok{1}\NormalTok{)}
\NormalTok{A }\OtherTok{\textless{}{-}} \FunctionTok{matrix}\NormalTok{(}\FunctionTok{rnorm}\NormalTok{(}\DecValTok{20}\NormalTok{), }\AttributeTok{ncol =} \DecValTok{2}\NormalTok{)}
\NormalTok{B }\OtherTok{\textless{}{-}} \FunctionTok{matrix}\NormalTok{(}\FunctionTok{rnorm}\NormalTok{(}\DecValTok{20}\NormalTok{), }\AttributeTok{ncol =} \DecValTok{2}\NormalTok{)}
\end{Highlighting}
\end{Shaded}

\begin{enumerate}
\def\labelenumi{\arabic{enumi}.}
\tightlist
\item
  What are the dimensions of \(A\) and \(B\)? Compute \(A^TB\) and
  \(AB^T\).
\item
  Combine \(A\) and \(B\) row-wise to create \(C\).
\item
  Let \(D\) be a copy of \(C\) centered around the mean columnwise. The
  unbiased estimator of the covariance matrix of \(C\) is defined as
  \[\frac{1}{n-1}D^TD,\] where \(n\) is the number of rows of \(D\).
  Compute this quantity. Compare with \texttt{cov(C)}.
\end{enumerate}

\hypertarget{lecture-5-control-structures}{%
\subsection{Lecture \#5: Control
Structures}\label{lecture-5-control-structures}}

\hypertarget{bootstrap}{%
\paragraph{Bootstrap}\label{bootstrap}}

The bootstrap is a well-known method in statistics since Efron's seminal
paper in 1979. The bootstrap is easy to implement and straightforward to
use. There exist many different schemes for the bootstrap, we present
the simplest form:

\begin{enumerate}
\def\labelenumi{\arabic{enumi}.}
\item
  Compute the statistic on the sample:
  \(\hat{\theta} = g(x_1,\dots,x_n)\).
\item
  Create a new sample \(x_1^\ast,\dots,x_n^\ast\) by drawing data from
  the original sample \textbf{at random with replacement}. This new
  sample is called a \emph{bootstrapped sample}.
\item
  Compute the statistic on the bootstrapped sample:
  \(\hat{\theta}^\ast = g(x_1^\ast,\dots,x_n^\ast)\).
\item
  Repeat 2. and 3. \(B\) times.
\item
  Compute the unbiased estimator of the variance:
  \[\frac{1}{B-1}\sum_{b=1}(\hat{\theta}^\ast_{b}-\hat{\theta})^2.\]
\item
  Load a dataset using \texttt{data("ToothGrowth")}. Create two vectors
  of tooth lengths corresponding to \texttt{OJ} and \texttt{VC} factors
  respectively. Compute the mean of each vector.
\item
  Create a bootstrap distribution for each vector using \(B=10,000\) and
  a for loop. Checkout the \texttt{sample} function for sampling at
  random with replacement.
\item
  Using \texttt{ggplot2}, make a graph of the bootstrap distributions by
  plotting two histograms on the same plot.
\end{enumerate}

\hypertarget{lecture-6-functions}{%
\subsection{Lecture \#6: Functions}\label{lecture-6-functions}}

\begin{itemize}
\tightlist
\item
  What does the following code return?
\end{itemize}

\begin{Shaded}
\begin{Highlighting}[]
\NormalTok{x }\OtherTok{\textless{}{-}} \DecValTok{2}
\NormalTok{f1 }\OtherTok{\textless{}{-}} \ControlFlowTok{function}\NormalTok{(x) \{}
  \ControlFlowTok{function}\NormalTok{() \{}
\NormalTok{    x }\SpecialCharTok{+} \DecValTok{3}
\NormalTok{  \}}
\NormalTok{\}}
\FunctionTok{f1}\NormalTok{(}\DecValTok{1}\NormalTok{)()}
\end{Highlighting}
\end{Shaded}

\begin{itemize}
\tightlist
\item
  How would you usually write these codes?
\end{itemize}

\begin{Shaded}
\begin{Highlighting}[]
\StringTok{\textasciigrave{}}\AttributeTok{+}\StringTok{\textasciigrave{}}\NormalTok{(}\DecValTok{1}\NormalTok{, }\StringTok{\textasciigrave{}}\AttributeTok{*}\StringTok{\textasciigrave{}}\NormalTok{(}\DecValTok{2}\NormalTok{, }\DecValTok{3}\NormalTok{))}
\StringTok{\textasciigrave{}}\AttributeTok{*}\StringTok{\textasciigrave{}}\NormalTok{(}\DecValTok{3}\NormalTok{, }\StringTok{\textasciigrave{}}\AttributeTok{+}\StringTok{\textasciigrave{}}\NormalTok{(}\DecValTok{2}\NormalTok{, }\DecValTok{1}\NormalTok{))}
\end{Highlighting}
\end{Shaded}

\begin{itemize}
\tightlist
\item
  How could you make this function call easier to read?
\end{itemize}

\begin{Shaded}
\begin{Highlighting}[]
\FunctionTok{mean}\NormalTok{(, }\ConstantTok{TRUE}\NormalTok{, }\AttributeTok{x =} \FunctionTok{c}\NormalTok{(}\FunctionTok{seq}\NormalTok{(}\DecValTok{10}\NormalTok{), }\FunctionTok{rep}\NormalTok{(}\ConstantTok{NA}\NormalTok{,}\DecValTok{3}\NormalTok{)))}
\end{Highlighting}
\end{Shaded}

\begin{itemize}
\tightlist
\item
  Does the following code throw an error when executed? Why or why not?
\end{itemize}

\begin{Shaded}
\begin{Highlighting}[]
\NormalTok{f2 }\OtherTok{\textless{}{-}} \ControlFlowTok{function}\NormalTok{(a, b) \{}
\NormalTok{  a }\SpecialCharTok{*} \DecValTok{3}
\NormalTok{\}}
\FunctionTok{f2}\NormalTok{(}\DecValTok{3}\NormalTok{, }\FunctionTok{stop}\NormalTok{(}\StringTok{"This is an error!"}\NormalTok{))}
\FunctionTok{f2}\NormalTok{(}\FunctionTok{stop}\NormalTok{(}\StringTok{"This is an error!"}\NormalTok{), }\DecValTok{3}\NormalTok{)}
\end{Highlighting}
\end{Shaded}

\begin{itemize}
\tightlist
\item
  Propose an infix function.
\end{itemize}

\end{document}
